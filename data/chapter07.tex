\section{滚动轴承的选择和计算}
\subsection{高速轴轴承的计算与校核}
(1)选用角接触球轴承,型号$7207AC$,尺寸$d\times D\times B=35\times 72\times 17$,基本额定动载荷$Cr=30.5kN$。轴承采用正装方式。

(2)轴承存在轴向载荷$F_a=1171.23N$,合成支撑反力$F_1=2162.47N,F_2=2224.6N$。
由轴承的内部轴向力公式
\[
    F_s=eF
\]
对角接触球轴承$e=0.68$。计算得到$F_{s1}=0.68\times 2162.47=1470.5N,F_{s2}=1512.73N$,考虑到轴向力的方向可以朝两个方向,假定轴向力从$F_1$指向$F_2$。
因此上得到
\[
    F_a+F_{s1}>F_{s2}
\]
此情况下,2端压紧$F_{a2}=F_a+F_{s1}=2641.7N$,1端放松$F_{a1}=F_{s1}=1470.5N$。
计算轴承的当量动载荷
\[
    \frac{F_{a1}}{F_1}=0.68
\]
\[
    \frac{F_{a2}}{F_2}=1.18
\]
查表得径向动载荷系数$X$和轴向动载荷系数分别为$Y$:
$X_1=1,Y_1=0;X_2=0.41,Y_2=0.87$
\begin{align}
    P_1=X_1 F_1+Y_1 F_{a1}=2162.46N\\
    P_2=X_2 F_2+Y_2 F_{a2}=3210.3N
\end{align}

齿轮受轻微冲击选择$f_P=1.1$,齿轮额定寿命$L_h=10000h$。
带入公式
\begin{equation}
    C_{r2}=\frac{f_P P_1}{f_1}(\frac{60n}{10^6} L_h)^{\frac{1}{3}}=29.53kN<[Cr]
\end{equation}
当其轴向力反向时计算得到
\[
    C_{r2}=22.02kN<[c_r]
\]
所以轴承预期寿命足够。

\subsection{低速轴轴承的计算与校核}
(1)选用角接触球轴承,型号$7012AC$,尺寸$d\times D\times B=60\times 95\times 18$,基本额定动载荷$Cr=36.2kN$。轴承采用正装方式。

(2)轴承存在轴向载荷$F_a=1115.77N$,合成支撑反力$F_1=4146.16N,F_2=1192.17N$。
由轴承的内部轴向力公式
\[
    F_s=eF
\]
对角接触球轴承$e=0.68$。计算得到$F_{s1}=0.68\times 4146.16=2819.39N,F_{s2}=810.67N$,考虑到轴向力的方向可以朝两个方向,假定轴向力从$F_1$指向$F_2$。
因此上得到
\[
    F_a+F_{s1}>F_{s2}
\]
此情况下,2端压紧$F_{a2}=F_a+F_{s1}=3935.16N$,1端放松$F_{a1}=F_{s1}=2819.39N$。
计算轴承的当量动载荷
\[
    \frac{F_{a1}}{F_1}=0.68
\]
\[
    \frac{F_{a2}}{F_2}=3.3
\]
查表得径向动载荷系数$X$和轴向动载荷系数分别为$Y$:
$X_1=1,Y_1=0;X_2=0.41,Y_2=0.87$
\begin{align}
    P_1=X_1 F_1+Y_1 F_{a1}=4146.16N\\
    P_2=X_2 F_2+Y_2 F_{a2}=3912.38N
\end{align}

齿轮受轻微冲击选择$f_P=1.1$,齿轮额定寿命$L_h=10000h$。
带入公式
\begin{equation}
    C_{r2}=\frac{f_P P_1}{f_1}(\frac{60n}{10^6} L_h)^{\frac{1}{3}}=22.61kN<[Cr]
\end{equation}
当其轴向力反向时计算得到
\[
    C_{r2}=22.61kN<[c_r]
\]
所以轴承预期寿命足够。