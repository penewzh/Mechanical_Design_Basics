\section{润滑与密封的选择,润滑剂牌号}
\subsection{轴承的润滑}
计算出高速轴和低速轴的速度因数$dn$分别等于$34125$和$12189$,其$dn<(2\sim 3)\times10^5mm\cdot r/min$,此时滚动轴承采用润滑脂润滑。

因此选用通用锂基润滑脂$(GB/T7324-2010)$,代号为$1$号,它适用于较宽温度范围的机械设备的润滑。

同时为了避免油脂被稀释,轴承与箱体内壁保持一定的距离且用挡油环将轴承与箱体隔开。
\subsection{齿轮的润滑}
计算出高速轴的齿轮转动速度
\[
    v= \pi dn=975r/min\times \pi \times 50\times 10^{-3} =5.1m/s
\]
低速轴的齿轮转动速度
\[
    v= \pi dn=203.15r/min\times \pi \times 240.4\times 10^{-3}=2.5m/s
\]
采用浸油润滑。浸油深度为0.7个齿高,同时不小于$10mm$。为了避免齿轮转动时将沉积在油池底部的污物搅起造成齿面磨损,大齿轮顶部距离箱体底面的距离$30mm$。大齿轮全尺高$h=2.23m_n=4.5mm$,取浸油深度为$10mm$,则油的深度为$40mm$。

根据齿轮圆周速度选用工业闭式齿轮油$(GB/T5903-2011)$,代号为$L-CKC320$。
\subsection{减速器的密封}
为防止箱体内润滑剂外泄和外部杂质进入箱体内部影响箱体工作,在构成箱体的各零件间,如箱盖与箱座间、外伸轴的输入、输出轴与轴承盖间,需设置不同形式的密封装置。对于无相对运动的接合面,常用密封胶、耐油橡胶垫圈等;对于旋转零件如外伸轴的密封,则需要根据其不同运动速度和密封要求考虑不同的密封件和结构。本设计中由于密封界面的相对速度较小,故采用接触式密封。输入轴与轴承盖间、输出轴与轴承盖间速度皆小于$3m/s$,故均采用半粗羊毛毡密封油圈。