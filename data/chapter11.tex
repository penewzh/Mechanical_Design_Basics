\section{设计小结}
课程设计是理论联系实际的一个过程。虽然只有短短的两个星期的时间,使我对机械设计这门课有了更深的理解和认识。机械设计的每一步都有它的理论支持,必须要按照它的标准进行。

作为一门基础学科,机械设计用到了之前学过的如工程图学、理论力学、材料力学、机械设计基础的内容。同时由于计算机技术的快速发展,计算机辅助设计也成为了机械设计工程不可缺少的技术。这次的机械设计中,我又通过复习了工程图学所讲述的CAD建模的知识,完成了零件和装配图的设计\cite{刘苏}。

最后,这次课程设计也离不开老师的悉心指导与帮助。在此衷心的感谢老师的指导与帮助。同时,由于时间短暂、知识的了解也不够充分,设计中会有一些不可避免的错误以及设计中未曾考虑到的缺点。如斜齿轮在正反转的过程中会有不同方向的轴向力,如何减少以至避免这个问题,还需要今后进一步的学习与思考,继续培养设计习惯和思维来提高实地操作的能力。